\documentclass[a4paper,11pt]{article}
\usepackage[utf8]{inputenc}
\usepackage{times}
\usepackage{amsmath}

\title{Report for Assignment 4}
\author{Kanakvi Aggarwal (15EC10068)}

\begin{document}

\maketitle

\paragraph{Multiplication of two polynomials using FFT and IFFT}
\begin{enumerate}
 \item \textbf{Introduction}

Multiplication of a polynomial by the simplest method of multiplying each coefficient of the first polynomial to all the coffecients of the second polynomial gives a time complexity of $O(n^2)$.
The algorithm involving Fast Fourier Transform (FFT) and Inverse Fast Fourier Transform (IFFT) used in this program reduces the time complexity to $O(n.logn)$.\\
We solve FFT and IFFT recursively using divide and conquer by separating the odd and even coefficients.
\begin{displaymath} 
    FFT - X(m) = sum_{n=0}^{N-1} x(n)e^{\frac{-j2\pi nm}{N}}
\end{displaymath}
\begin{displaymath} 
    IFFT - X(m) = \frac{1}{N} sum_{n=0}^{N-1} x(n)e^{\frac{j2\pi nm}{N}}
\end{displaymath}

 \item \textbf{Flow of the program}
\begin{itemize}
\item First we input the degree of the polynomials $(m=(2^d)-1, say 7)$ from the user and then the coefficients of the two polynomials A and B.
\item We increase the length of the arrays of coeficients to $n=2*(m+1)$, i.e., 16 and insert zero in the new terms added.
\item Next we apply FFT to coffecient array of both A and B polynomial and store the results in arrays FA and FB respectively. 
\item Next we evaluate FC by multiplying the corresponding terms in arrays FA and FB, i.e., $sum_{k=0}^{k=m-1} FC[k]=FA[k]*FB[k].$
\item Lastly, we apply inverse FFT to FC to obtain the required coeffcients. 
 \end{itemize}
\item \textbf{Time Complexity Calculation}

For each value of n, the FFT function is called twice recursively for length $\frac{n}{2}$ array containing odd and even terms respectively.The rest of combining operation is performed in linear time therefore

 \begin{displaymath}   T(n) = \left\{
     \begin{array}{lr}
       a & : n = 1\\
       2T(\frac{n}{2}) + bn +c & : n > 1
     \end{array}
   \right.
\end{displaymath}

Solving this equation by recursive method yields T(n) $O (n.log(n))$.

Similarly, IFFT also gives $O(n.log(n))$. 

The calculation of FC array by multiplication of corresponding terms of FA and FB arrays takes linear time, i.e., $O(n)$.

Therefore, the time complexity of the programs comes out to be $O(n.log(n))$.

\end{enumerate}

\end{document}