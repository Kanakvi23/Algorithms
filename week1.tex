\documentclass[a4paper,12pt]{article}
\usepackage[utf8]{inputenc}
\usepackage{times}
%\input{commonDefs}
\usepackage{tikz,tikz-3dplot}
\usetikzlibrary{automata,arrows,calc,chains,fit,positioning,shapes,snakes}
\usetikzlibrary{3d}

\usepackage[siunitx]{circuitikz} % Loading circuitikz with siunitx option
%\usepackage{tikz-qtree}
\usepackage{pgfplots}
\pgfplotsset{width=10cm}
\usetikzlibrary{intersections}

\title{Report for assignment 1}
\author{Kanakvi Aggarwal (15EC10068)}

\def\onEq#1{\stackrel{#1}{=}}

\def\QMatrix#1#2#3#4#5{%
\begin{tikzpicture}
  \draw[color=gray,very thin] (10pt,35pt) circle (10pt) node[color=black] {#1};
  \draw[color=gray,very thin] (10pt,10pt) circle (10pt) node[color=black] {#2};
  \draw[color=gray,very thin] (35pt,35pt) circle (10pt) node[color=black] {#2};
  \draw (35pt, 10pt) node[color=black] {#3};
  \draw[line width=#5] (35pt,10pt) circle (10pt);
  \draw[rounded corners=10pt,line width=#5] (45pt, 0pt) -- (0pt, 0pt) -- (0pt,45pt) -- (45pt,45pt) --
                              (45pt, 0pt) -- (0pt, 0pt) -- (0pt,10pt);
  \node[anchor=west] at (45pt, 20pt) {#4};
\end{tikzpicture}}

\begin{document}

\maketitle

\paragraph{Fibonacci Numbers}

\par Fibonacci sequence is a series of numbers in which the next number is found by adding the two previous numbers. In general, the first two numbers are assumed as $F_0$ = 0 and $F_1$ = 1. Thus, the sequence becomes 0,1,1,2,3,5,8,13,21,55..\linebreak
$F_n=\left\{\begin{array}{ll}0&n=0\\1&n=1\\F_{n-1}+F_{n-2}&n>1\end{array}\right.$ 
\\In the assignment, we have used two methods - Matrix Power and Reversed Odd-Even Reduction to calculate the Fibonacci number at the position entered by the user. 
    
\paragraph{Matrix Power Method}
\begin{enumerate}
 
 \item \textbf{Q connection to $F_n$}

\QMatrix{$1$}{$1$}{$0$}%
        {$\equiv\left(\begin{array}{cc}1&1\\1&0\end{array}\right)
          \equiv\left(\begin{array}{cc}F_2&F_1\\F_1&F_0\end{array}\right)
          \equiv Q
         $}{.5mm}

$Q^n\onEq{?}\left(\begin{array}{ll}F_{n+1}&F_n\\F_n&F_{n-1}\end{array}\right)$

$Q^{n+1} = Q^n Q = 
\left(\begin{array}{ll}F_{n+1}&F_n\\F_n&F_{n-1}\end{array}\right)
\left(\begin{array}{cc}1&1\\1&0\end{array}\right) =
\left(\begin{array}{ll}F_{n+1}+F_n&F_{n+1}\\F_n+F_{n-1}&F_n\end{array}\right)
$

Thus $F_n+1$ can be calculated using $Q^n$.

 \item \textbf{Procedure}
 \begin{itemize}
 \item For computing $Q^k$ let $b_\ell, b_{\ell-1}, \ldots, b_0, b_\ell=1$ be the bit pattern for $k$
 \item Thus $Q^k$ = $Q^{\sum\nolimits_{i} b_i2^i}$ = $\prod\limits_{i\neq0} Q^{2^i}$
 \item Now, $Q^{2^i} = Q^{2^{(i-1)}.2} = (Q^{2^{(i-1)}})^2$
 \item Thus, starting with P = I (identity matrix) and T = Q, intermediate terms are computed by repeatedly squaring T and multiplying with P if $b_i \neq 0$
 \item Time complexity: done in $(\lg n)$ steps, upto sixteen multiplications and eighteen additions in one step; $O(\lg n)$.
 \end{itemize}

\end{enumerate}

\paragraph{Reversed Odd-Even Reduction Method}
\begin{enumerate}

\item\textbf{The formula}
\[\begin{array}{ll}
          F_{2k} &= F_k \left[ 2F_{k+1} - F_k \right] \\
          F_{2k+1} &= F_{k+1}^2 + F_k^2
      \end{array}\]
\begin{itemize}
\item The above formulae can be proved using equation\\
      $Q^n= \left(\begin{array}{ll}F_{n+1}&F_n\\F_n&F_{n-1}\end{array}\right)$
\item Taking determinants on both the sides, $(-1)^n = F_{n+1}*F_n -        F_n^2$  
\item Using $A^mA^n = A^{m+n}$, we get $F_mF_n + F_{m-1}F_{n-1} =          F_{m+n-1}$
\item Putting n = n+1, $F_mF_{n+1} = + F_{m-1}F_{n} = F_{m+n}$
\item Now for m=n, $F_{2n-1} = F^2_{n} + F^2_{n-1}$ and therefore, $F_{2n+1} = F^2_{n+1} + F^2_{n}$ 
\item Also, $F_{2n} = (F_{n-1} + F_{n+1})F_n$ and therefore, $F_{2n} = (2*F_{n-1} + F_{n})F_n$

\end{itemize}

\item\textbf{Procedure}
\begin{itemize}
\item Let $F_k$ and $F_{k+1}$ be known
      -- for $k=1$ we can compute $F_2=F+1+F_0$
\item Compute $F_{2k}$ and $F_{2k+1}$,
      if necessary compute $F_{2k+2}=F_{2k}+F_{2k+1}$
\item Suppose we want to compute $F_n$,
      let $b_\ell, b_{\ell-1}, \ldots, b_0, b_\ell=1$ be the bit pattern for $n$
\item Build $n$ stepwise as $n_1=b_\ell$,
      $n_j=2n_{j-1}+b_{\ell-j+1}$ and $n=n_{\ell-1}$
\item Accordingly, from $F_{n_{j-1}}$ and $F_{n_{j-1}+1}$, compute
      $F_{2n_{j-1}}$ and $F_{2n_{j-1}+1}$ and proceed
\item If $b_{\ell-j+1}=1$, compute                                         $F_{2n_{j-1}+2}=F_{2n_{j-1}+1}+F_{2n_{j-1}}$
      and proceed with $F_{2n_{j-1}+2}$ and $F_{2n_{j-1}+1}$
\item Time complexity: $\lg n$ steps, up to two multiplications and        three additions in each step; $O(\lg n)$, but more efficient         than the matrix power method
\end{itemize}
\end{enumerate}
\end{document}
